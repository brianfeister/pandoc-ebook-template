\documentclass[12pt]{report}

\usepackage[
  letterpaper,
  margin=0.5in,
]{geometry}

% General spacing settings

%\usepackage{setspace}
%\setstretch{}

%\setlength{\parindent}{0pt}
%\setlength{\parskip}{6pt plus 2pt minus 1pt}
%\setlength{\emergencystretch}{3em}  % prevent overfull lines

% IMPORTANT: remember to use this if lists are present
%\providecommand{\tightlist}{%
%  \setlength{\itemsep}{0pt}\setlength{\parskip}{0pt}}

% Math extensions

%\usepackage{amssymb,amsmath}

% Font selection

\usepackage{fontspec}
\setmainfont{Droid Serif}
\setsansfont{Alegreya SC}
\setmonofont{Alegreya SC}

% Hyperlinks

\usepackage[unicode=true]{hyperref}

\hypersetup{
  % this will need a bit of cleanup
  breaklinks=true,
  bookmarks=true,
  pdfauthor={J. Random Author},
  pdftitle={My Magnum Opus},
  colorlinks=true,
  citecolor=blue,
  urlcolor=blue,
  linkcolor=magenta,
  pdfborder={0 0 0},
}

% don't use monospace font for urls
\urlstyle{same}

% Page headers/footers

\usepackage{fancyhdr}
\pagestyle{fancy}
\pagenumbering{arabic}
\lhead{\itshape My Magnum Opus}
\chead{}
\rhead{\itshape{\nouppercase{\leftmark}}}
\lfoot{v }
\cfoot{}
\rfoot{\thepage}

% Language selection

\usepackage{polyglossia}
\setmainlanguage{english}

% Table packages

\usepackage{longtable,multicol,booktabs}

% Graphics

%
% Ugly paragraph hack

%\ifx\paragraph\undefined\else
%\let\oldparagraph\paragraph
%\renewcommand{\paragraph}[1]{\oldparagraph{#1}\mbox{}}
%\fi
%\ifx\subparagraph\undefined\else
%\let\oldsubparagraph\subparagraph
%\renewcommand{\subparagraph}[1]{\oldsubparagraph{#1}\mbox{}}
%\fi

% Background image

\usepackage{eso-pic}
\usepackage{tikz}
\usetikzlibrary{positioning}

\AddToShipoutPicture{
\tikz [overlay, remember picture] \node at (current page.center) {\includegraphics[width=\paperwidth,height=\paperheight,keepaspectratio]{assets/paper-bg.jpg}};
}

% Microtype

\usepackage{microtype}

% Main document

\title{My Magnum Opus}

\author{J. Random Author}

\date{}

\begin{document}
\maketitle
{
\hypersetup{linkcolor=black}
\setcounter{tocdepth}{2}
\tableofcontents
}
%\listoftables
%\listoffigures
\chapter{Chapter 1: Character
Creation}\label{chapter-1-character-creation}

Before you can start telling your story, you'll need a character to
play. This chapter will offer you step-by-step instructions to creating
your own hero. In \emph{Open Legend}, you typically begin as a level one
character. As you complete quests and gain more experience adventuring,
you'll level up and gain more power. These rules explain how to create a
character starting at level one. Later, you'll learn what to do when you
level up.

Before reading on, take a minute and think of your favorite fantasy
movies, books, or video games.

\emph{Who were the characters you identified with?}

\emph{Who inspired you?}

Now that you've got some of your favorites in mind, let's create your
character.

\subsection{Choose an Example Character \& Modify
It}\label{choose-an-example-character-modify-it}

For anyone who feels they could benefit from some inspiration, you can
easily make a copy of any of these spreadsheets (they include formulas
for doing most all of the calculations for you), and then increase or
decrease attribute scores, as well as add or remove feats. Be sure to
adjust the \textbf{Level} field to get the correct calculations for
attribute \& feat points at your current character level.

\href{https://drive.google.com/drive/u/0/folders/0Bx_UrXHMi3wmUlJjbDZiaGtIX00}{\textbf{View
Pre-generated Character Options}}

\section{Step 1: Choose Attributes}\label{step-1-choose-attributes}

Attributes are the backbone of every character in \emph{Open Legend}.
They define what your character can and can't do--the spheres he excels
in, as well as his greatest weaknesses. Whenever your character attempts
a heroic action in Open Legend, you'll look to your attributes to see
how well you succeed or fail.

In \emph{Open Legend}, attributes are divided into four categories:
physical, social, mental, and supernatural.

A character's skill with each attribute is expressed as a score from 0
(completely unpracticed) to 9 (superhuman). The average commoner or
craftsmen usually has scores ranging from 1 - 3 in several physical,
social, and mental attributes. Supernatural attributes are generally
reserved for characters of power and note.

The Attributes at a Glance tables provide a quick overview of some of
the common actions that each attribute will help you accomplish.

\begin{center}\rule{0.5\linewidth}{\linethickness}\end{center}

\textbf{Physical Attributes at a Glance}

\begin{longtable}[c]{@{}ll@{}}
\toprule
\begin{minipage}[t]{0.03\columnwidth}\raggedright\strut
\textbf{Agility}
\strut\end{minipage} &
\begin{minipage}[t]{0.07\columnwidth}\raggedright\strut
Dodge attacks, move with stealth, perform acrobatics, shoot a bow, pick
a pocket
\strut\end{minipage}\tabularnewline
\begin{minipage}[t]{0.03\columnwidth}\raggedright\strut
\textbf{Fortitude}
\strut\end{minipage} &
\begin{minipage}[t]{0.07\columnwidth}\raggedright\strut
Wear heavy armor, resist poison, shrug off pain, exert yourself
physically
\strut\end{minipage}\tabularnewline
\begin{minipage}[t]{0.03\columnwidth}\raggedright\strut
\textbf{Might}
\strut\end{minipage} &
\begin{minipage}[t]{0.07\columnwidth}\raggedright\strut
Swing a maul, jump over a chasm, break down a door, wrestle a foe to
submission
\strut\end{minipage}\tabularnewline
\bottomrule
\end{longtable}

\begin{longtable}[c]{@{}l@{}}
\toprule
Mental Attributes at a Glance\tabularnewline
\midrule
\endhead
\tabularnewline
\bottomrule
\end{longtable}

\begin{longtable}[c]{@{}ll@{}}
\toprule
\begin{minipage}[t]{0.03\columnwidth}\raggedright\strut
\textbf{Learning}
\strut\end{minipage} &
\begin{minipage}[t]{0.03\columnwidth}\raggedright\strut
Recall facts about history, arcane magic, the natural world, etc.
\strut\end{minipage}\tabularnewline
\begin{minipage}[t]{0.03\columnwidth}\raggedright\strut
\textbf{Logic}
\strut\end{minipage} &
\begin{minipage}[t]{0.03\columnwidth}\raggedright\strut
Solve riddles, decipher a code, improvise a tool, understand the enemy's
strategy, find a loophole
\strut\end{minipage}\tabularnewline
\begin{minipage}[t]{0.03\columnwidth}\raggedright\strut
\textbf{Perception}
\strut\end{minipage} &
\begin{minipage}[t]{0.03\columnwidth}\raggedright\strut
Sense ulterior motives, track someone, catch a gut feeling, spot a
hidden foe, find a secret door
\strut\end{minipage}\tabularnewline
\begin{minipage}[t]{0.03\columnwidth}\raggedright\strut
\textbf{Will}
\strut\end{minipage} &
\begin{minipage}[t]{0.03\columnwidth}\raggedright\strut
Maintain your resolve, overcome adversity, resist torture, stay awake on
watch, stave off insanity
\strut\end{minipage}\tabularnewline
\bottomrule
\end{longtable}

\begin{longtable}[c]{@{}l@{}}
\toprule
Social Attributes at a Glance\tabularnewline
\midrule
\endhead
\tabularnewline
\bottomrule
\end{longtable}

\begin{longtable}[c]{@{}ll@{}}
\toprule
\begin{minipage}[t]{0.03\columnwidth}\raggedright\strut
\textbf{Deception}
\strut\end{minipage} &
\begin{minipage}[t]{0.03\columnwidth}\raggedright\strut
Tell a lie, bluff at cards, disguise yourself, spread rumors, swindle a
sucker
\strut\end{minipage}\tabularnewline
\begin{minipage}[t]{0.03\columnwidth}\raggedright\strut
\textbf{Persuasion}
\strut\end{minipage} &
\begin{minipage}[t]{0.03\columnwidth}\raggedright\strut
Negotiate a deal, convince someone, haggle a good price, pry information
\strut\end{minipage}\tabularnewline
\begin{minipage}[t]{0.03\columnwidth}\raggedright\strut
\textbf{Presence}
\strut\end{minipage} &
\begin{minipage}[t]{0.03\columnwidth}\raggedright\strut
Give a speech, sing a song, inspire an army, exert your force of
personality, have luck smile upon you
\strut\end{minipage}\tabularnewline
\bottomrule
\end{longtable}

\begin{longtable}[c]{@{}l@{}}
\toprule
Supernatural Attributes at a Glance\tabularnewline
\midrule
\endhead
\tabularnewline
\bottomrule
\end{longtable}

\begin{longtable}[c]{@{}ll@{}}
\toprule
\begin{minipage}[t]{0.03\columnwidth}\raggedright\strut
\textbf{Abjuration}
\strut\end{minipage} &
\begin{minipage}[t]{0.03\columnwidth}\raggedright\strut
Protect from damage, break enchantments, dispel magic, bind demons
\strut\end{minipage}\tabularnewline
\begin{minipage}[t]{0.03\columnwidth}\raggedright\strut
\textbf{Entropy}
\strut\end{minipage} &
\begin{minipage}[t]{0.03\columnwidth}\raggedright\strut
Disintegrate matter, kill with a word, create undead, sicken others
\strut\end{minipage}\tabularnewline
\begin{minipage}[t]{0.03\columnwidth}\raggedright\strut
\textbf{Alteration}
\strut\end{minipage} &
\begin{minipage}[t]{0.03\columnwidth}\raggedright\strut
Change shape, alter molecular structures, transmute one material into
another
\strut\end{minipage}\tabularnewline
\begin{minipage}[t]{0.03\columnwidth}\raggedright\strut
\textbf{Enchantment}
\strut\end{minipage} &
\begin{minipage}[t]{0.03\columnwidth}\raggedright\strut
Control the minds of others, dominate wills, speak telepathically,
instill supernatural fear
\strut\end{minipage}\tabularnewline
\begin{minipage}[t]{0.03\columnwidth}\raggedright\strut
\textbf{Creation}
\strut\end{minipage} &
\begin{minipage}[t]{0.03\columnwidth}\raggedright\strut
Channeling higher powers for healing, creation, resurrection, divine
might, etc.
\strut\end{minipage}\tabularnewline
\begin{minipage}[t]{0.03\columnwidth}\raggedright\strut
\textbf{Illusion}
\strut\end{minipage} &
\begin{minipage}[t]{0.03\columnwidth}\raggedright\strut
Create illusory figments, deceive the senses, cloak with invisibility
\strut\end{minipage}\tabularnewline
\begin{minipage}[t]{0.03\columnwidth}\raggedright\strut
\textbf{Divination}
\strut\end{minipage} &
\begin{minipage}[t]{0.03\columnwidth}\raggedright\strut
See the future, detect magic, detect evil, scry, communicate with
extraplanar entities
\strut\end{minipage}\tabularnewline
\begin{minipage}[t]{0.03\columnwidth}\raggedright\strut
\textbf{Movement}
\strut\end{minipage} &
\begin{minipage}[t]{0.03\columnwidth}\raggedright\strut
Teleport, fly, hasten, slow
\strut\end{minipage}\tabularnewline
\begin{minipage}[t]{0.03\columnwidth}\raggedright\strut
\textbf{Energy}
\strut\end{minipage} &
\begin{minipage}[t]{0.03\columnwidth}\raggedright\strut
Create and control the elements--fire, cold, electricity, etc.
\strut\end{minipage}\tabularnewline
\begin{minipage}[t]{0.03\columnwidth}\raggedright\strut
\textbf{Psychic}
\strut\end{minipage} &
\begin{minipage}[t]{0.03\columnwidth}\raggedright\strut
Psychokinesis, telekinesis, mind over matter, extrasensory perception
\strut\end{minipage}\tabularnewline
\bottomrule
\end{longtable}

In \emph{Open Legend}, you get to define your character's strengths and
weaknesses by choosing the attributes that fit your character concept.
Described below are several methods by which you can assign your
attributes.

\subsection{Quick Build}\label{quick-build}

If you are new to roleplaying games, or are just looking to get your
character built quickly, choose one of the attribute sets listed in the
Attribute Quick Builds table. Assign the scores listed to the attributes
that define the type of character you want to play. The rest of your
attributes begin with a score of zero.

\begin{longtable}[c]{@{}l@{}}
\toprule
ATTRIBUTE QUICK BUILDS\tabularnewline
\midrule
\endhead
\textbf{Specialized Hero}\tabularnewline
5, 4, 3, 2, 2, 2\tabularnewline
\textbf{Well-rounded Hero}\tabularnewline
4, 4, 3, 3, 3, 1\tabularnewline
\textbf{Jack of All Trades}\tabularnewline
3, 3, 3, 3, 2, 2, 2, 2, 1\tabularnewline
\bottomrule
\end{longtable}

\subsection{Custom Build}\label{custom-build}

If you would like more control over your attributes, you can purchase
them to create your own set. With this method, at first level, you have
a budget of 40 attribute points to spend, and the cost of each score is
defined in the Purchasing Attributes table. The highest any score can
reach at first level is 5, and you don't have to spend all of your
points at character creation.

\begin{longtable}[c]{@{}l@{}}
\toprule
PURCHASING ATTRIBUTES\tabularnewline
\midrule
\endhead
\tabularnewline
\bottomrule
\end{longtable}

\begin{longtable}[c]{@{}ll@{}}
\toprule
Attribute Score & Cost\tabularnewline
\midrule
\endhead
0 & 0\tabularnewline
1 & 1\tabularnewline
2 & 3\tabularnewline
3 & 6\tabularnewline
4 & 10\tabularnewline
5 & 15\tabularnewline
\bottomrule
\end{longtable}

\subsection{Archetype Build}\label{archetype-build}

In the Archetype Attribute Builds table, several common fantasy
archetypes are listed. If you are envisioning a character similar to one
of these, you can just take one of these sets as written, or swap out
some of the attributes for others.

\begin{longtable}[c]{@{}l@{}}
\toprule
ARCHETYPE ATTRIBUTE BUILDS\tabularnewline
\midrule
\endhead
\tabularnewline
\bottomrule
\end{longtable}

\begin{longtable}[c]{@{}lll@{}}
\toprule
Barbarian & Ranger & Monk\tabularnewline
\midrule
\endhead
Agility 2 & Agility 5 & Agility 5\tabularnewline
Fortitude 4 & Deception 2 & Fortitude 2\tabularnewline
Might 5 & Perception 4 & Logic 1\tabularnewline
Perception 3 & Will 3 & Perception 3\tabularnewline
Will 3 & Enchantment 3 & Will 2\tabularnewline
& & Psychic 4\tabularnewline
\bottomrule
\end{longtable}

\begin{longtable}[c]{@{}lll@{}}
\toprule
Paladin & Battle Mage & Mind Mage\tabularnewline
\midrule
\endhead
Fortitude 4 & Agility 3 & Agility 3\tabularnewline
Presence 5 & Fortitude 1 & Presence 2\tabularnewline
Learning 1 & Might 1 & Persuasion 1\tabularnewline
Perception 1 & Presence 1 & Learning 2\tabularnewline
Will 3 & Persuasion 1 & Logic 2\tabularnewline
Creation 3 & Learning 2 & Will 2\tabularnewline
Divination 1 & Logic 2 & Enchantment 5\tabularnewline
& Perception 2 & Illusion 3\tabularnewline
& Will 3 &\tabularnewline
& Energy 5 &\tabularnewline
\bottomrule
\end{longtable}

\begin{longtable}[c]{@{}lll@{}}
\toprule
Assassin & Cleric & Druid\tabularnewline
\midrule
\endhead
Agility 5 & Fortitude 2 & Agility 1\tabularnewline
Fortitude 1 & Might 3 & Fortitude 3\tabularnewline
Deception 3 & Presence 1 & Might 1\tabularnewline
Presence 3 & Persuasion 1 & Perception 2\tabularnewline
Persuasion 3 & Learning 1 & Will 2\tabularnewline
Perception 3 & Will 2 & Alteration 4\tabularnewline
Abjuration 4 & Creation 3 &\tabularnewline
Spirit 5 & Enchantment 4 &\tabularnewline
\bottomrule
\end{longtable}

\begin{longtable}[c]{@{}lll@{}}
\toprule
Shadowdancer & Bard & Arcane Protector\tabularnewline
\midrule
\endhead
Agility 5 & Agility 4 & Agility 3\tabularnewline
Deception 3 & Presence 4 & Presence 2\tabularnewline
Perception 3 & Persuasion 3 & Learning 3\tabularnewline
Will 1 & Learning 1 & Logic 1\tabularnewline
Entropy 3 & Creation 2 & Will 2\tabularnewline
Movement 3 & Enchantment 4 & Abjuration 5\tabularnewline
& Movement 3 &\tabularnewline
\bottomrule
\end{longtable}

\subsection{Record Attribute Dice}\label{record-attribute-dice}

Every attribute score above 0 grants you bonus dice to increase your
chance of success. Consult the Attribute Dice table for each of your
attributes and record the appropriate dice. (You'll learn what to do
with these dice later on.)

\begin{longtable}[c]{@{}l@{}}
\toprule
Attribute Dice\tabularnewline
\midrule
\endhead
\tabularnewline
\bottomrule
\end{longtable}

\begin{longtable}[c]{@{}ll@{}}
\toprule
Attribute Score & Attribute Dice\tabularnewline
\midrule
\endhead
1 & 1d4\tabularnewline
2 & 1d6\tabularnewline
3 & 1d8\tabularnewline
4 & 1d10\tabularnewline
5 & 2d6\tabularnewline
\bottomrule
\end{longtable}

\begin{quote}
\subsection{Roll Them Bones}\label{roll-them-bones}

If you are new to gaming, you may not be familiar with dice notation,
such as 2d6.

As you play Open Legend, you'll often need to roll dice to determine the
outcome of actions. \textbf{Dice notation} is a shorthand way of
indicating which dice to roll.

Every die roll is indicated by a formula such as 3d6. The number before
the \emph{d} indicates how many dice to roll, and the number after the
\emph{d} indicates how many sides those dice have.

So, 3d6 means that you roll three six-sided dice.

4d4 indicates four four-sided dice.

And so on.
\end{quote}

\section{Step 2: Calculate Defenses and Hit
Points}\label{step-2-calculate-defenses-and-hit-points}

When an enemy tries to attack you--whether with a breath of flame, a
deft sword thrust, or a mental assault--it will first need to overcome
your defense. You have three defense scores, and each one protects you
from different types of attacks. The higher your defense, the better you
are at avoiding or shrugging off whatever your foes throw at you.

\begin{longtable}[c]{@{}l@{}}
\toprule
Toughness = 10 + Fortitude\tabularnewline
\midrule
\endhead
\tabularnewline
\bottomrule
\end{longtable}

\textbf{Toughness} protects you from attacks that test your endurance,
bodily health, or sturdiness. For example, foes attempting to poison
you, push you off a cliff, or crush you with an iron maul will target
your toughness.

\begin{longtable}[c]{@{}l@{}}
\toprule
Evasion = 10 + Agility\tabularnewline
\midrule
\endhead
\tabularnewline
\bottomrule
\end{longtable}

\textbf{Evasion} protects you from attacks that test your quickness and
ability to dodge. Your enemies would need to overcome your evasion in
order to hit you with a fireball, shoot you with an arrow, or stab you
with a rapier.

\begin{longtable}[c]{@{}l@{}}
\toprule
Resolve = 10 + Presence + Will\tabularnewline
\midrule
\endhead
\tabularnewline
\bottomrule
\end{longtable}

\textbf{Resolve} represents your character's ability to resist mental
domination and stand brave in the face of danger. Enemies who wish to
magically charm you, deceive you with illusions, or frighten you must
target your resolve.

\begin{longtable}[c]{@{}l@{}}
\toprule
Hit Points = 2 x (Fortitude + Presence + Will) + 10\tabularnewline
\midrule
\endhead
\tabularnewline
\bottomrule
\end{longtable}

\emph{That is, add your Fortitude, Presence, and Will scores. Multiply
the total by 2. Finally, add 10.}

\textbf{Hit Points} (or HP) are an abstract measure of how well you can
ignore pain, avoid deadly blows, and maintain a presence on the
battlefield in spite of wounds or exhaustion. If they reach zero, you
fall unconscious and are at risk of death.

\section{Step 3: Purchase Feats}\label{step-3-purchase-feats}

While your character's attributes define his skill at accomplishing
heroic tasks, his \textbf{feats} are what make him unique among other
characters. Feats allow you to customize your character, granting him
the ability to accomplish specific actions exceptionally well.

For example, two different characters who specialize in melee combat
might both start with a Might score of 5. However, one character is a
swashbuckling pirate, so he takes the \emph{Combat Momentum} feat to
allow him to move deftly from one foe to the next like a whirling
dervish. The other, a battle-scarred berserker, takes \emph{Berserker}
so that she can fly into a frenzied rage in order to decimate her foes.

\subsection{Choose your feats}\label{choose-your-feats}

Feats are purchased using feat points. At 1st level, you have 6 feat
points to spend. Any leftover feat points may be saved for the future.

You can read the feat descriptions in a searchable database
\href{http://www.openlegendrpg.com/feats}{\emph{here}}.

If your character is based on a specific archetype, you may want to
start with the feat selections recommended below:

\begin{longtable}[c]{@{}l@{}}
\toprule
Archetype Feat Recommendations\tabularnewline
\midrule
\endhead
\tabularnewline
\bottomrule
\end{longtable}

\begin{longtable}[c]{@{}lll@{}}
\toprule
Barbarian & Ranger & Monk\tabularnewline
\midrule
\endhead
Berserker & Master Tracker & Fleet of Foot 1\tabularnewline
Reckless Frenzy & Attack Specialization 1 (Longbow) & Martial Focus
(Unarmed)\tabularnewline
& Multi-target Attack Specialist 1 (melee) & Combat
Momentum\tabularnewline
\bottomrule
\end{longtable}

\begin{longtable}[c]{@{}lll@{}}
\toprule
\begin{minipage}[b]{0.03\columnwidth}\raggedright\strut
Paladin
\strut\end{minipage} &
\begin{minipage}[b]{0.03\columnwidth}\raggedright\strut
Elemental Mage
\strut\end{minipage} &
\begin{minipage}[b]{0.03\columnwidth}\raggedright\strut
Mind Mage
\strut\end{minipage}\tabularnewline
\midrule
\endhead
\begin{minipage}[t]{0.03\columnwidth}\raggedright\strut
Attribute Substitution (Presence \textgreater{} Might)
\strut\end{minipage} &
\begin{minipage}[t]{0.03\columnwidth}\raggedright\strut
Area Manipulation 1
\strut\end{minipage} &
\begin{minipage}[t]{0.03\columnwidth}\raggedright\strut
Hallucination
\strut\end{minipage}\tabularnewline
\begin{minipage}[t]{0.03\columnwidth}\raggedright\strut
Armor Mastery (Scale Mail)
\strut\end{minipage} &
\begin{minipage}[t]{0.03\columnwidth}\raggedright\strut
Attack Specialization 1 (Cold)
\strut\end{minipage} &
\begin{minipage}[t]{0.03\columnwidth}\raggedright\strut
Potent Bane (Phantasm)
\strut\end{minipage}\tabularnewline
\begin{minipage}[t]{0.03\columnwidth}\raggedright\strut
Multi-target Attack Specialist (Area)
\strut\end{minipage} &
\begin{minipage}[t]{0.03\columnwidth}\raggedright\strut
\strut\end{minipage} &
\begin{minipage}[t]{0.03\columnwidth}\raggedright\strut
\strut\end{minipage}\tabularnewline
\bottomrule
\end{longtable}

\begin{longtable}[c]{@{}lll@{}}
\toprule
\begin{minipage}[b]{0.03\columnwidth}\raggedright\strut
Assassin
\strut\end{minipage} &
\begin{minipage}[b]{0.03\columnwidth}\raggedright\strut
Cleric
\strut\end{minipage} &
\begin{minipage}[b]{0.03\columnwidth}\raggedright\strut
Druid
\strut\end{minipage}\tabularnewline
\midrule
\endhead
\begin{minipage}[t]{0.03\columnwidth}\raggedright\strut
Martial Focus (Dagger)
\strut\end{minipage} &
\begin{minipage}[t]{0.03\columnwidth}\raggedright\strut
Restorative Adept 1
\strut\end{minipage} &
\begin{minipage}[t]{0.03\columnwidth}\raggedright\strut
Armor Specialization 2 (Scale Mail)
\strut\end{minipage}\tabularnewline
\begin{minipage}[t]{0.03\columnwidth}\raggedright\strut
Lethal Strike 1
\strut\end{minipage} &
\begin{minipage}[t]{0.03\columnwidth}\raggedright\strut
Armor Specialization 1 (Scale Mail)
\strut\end{minipage} &
\begin{minipage}[t]{0.03\columnwidth}\raggedright\strut
Ferocious Minions
\strut\end{minipage}\tabularnewline
\begin{minipage}[t]{0.03\columnwidth}\raggedright\strut
Master Shifter 1
\strut\end{minipage} &
\begin{minipage}[t]{0.03\columnwidth}\raggedright\strut
Boon Focus 1 (Shapeshift)
\strut\end{minipage} &
\begin{minipage}[t]{0.03\columnwidth}\raggedright\strut
\strut\end{minipage}\tabularnewline
\bottomrule
\end{longtable}

\begin{longtable}[c]{@{}lll@{}}
\toprule
Shadowdancer & Bard & Arcane Protector\tabularnewline
\midrule
\endhead
Lethal Strike 1 & Tactical Inspiration 2 & Defensive Expert
1\tabularnewline
Boon Focus 1 (Teleport) & Restorative Adept 1 & Boon Focus
(Teleport)\tabularnewline
\bottomrule
\end{longtable}

\section{Step 4: Choose Your Race}\label{step-4-choose-your-race}

While your character in \emph{Open Legend} can be a human, there are
also a number of other races to choose, from old standards like elves
and dwarves to the more obscure dragonbloods and celestials.

The standard races are described below, along with several defining
racial features typical to members of that race. Choose your character's
race, and then choose \textbf{one} of the listed racial features.

\subsection{Celestial}\label{celestial}

\textbf{Pure-hearted.} Any good-aligned creature you encounter is
friendly toward you by default rather than neutral. Circumstances can
alter this, but even if rumors or actions you've taken would influence a
good creature negatively, it remains one step friendlier than it
otherwise would have been.

\textbf{Divine Insight.} Drawing on a supernatural connection to a
deity, you gain otherworldly insight. Once per game session, you can
choose a topic relevant to the story. The GM shares some information
about that topic which might be useful. If you've just failed a
\emph{Learning} attribute roll and use this ability, the GM decides
whether to give you information related to that roll or to give you
knowledge that is completely unrelated.

\textbf{Trustworthy}. Once per session, you can choose to inspire trust
in another character via your good-hearted nature. That character
believes that you have their best interest in mind and seeks your advice
on something that is troubling them, sharing a secret that they would
not normally share with a stranger.

\textbf{Divine Intervention.} As a divine agent, you are defended by
your creator. Once per game session, when you are subject to a
\emph{Finishing Blow} while your hit points are below 1, you
automatically heal to a hit point total of 1.

\subsection{Dragonblood}\label{dragonblood}

\textbf{Legendary Bloodline}. Having dragon's blood, you command the
respect of those who practice the Arcane arts. You are assumed to have
knowledge and a destiny for greatness in the Arcane arts, and others
treat you with deference. This influence could guarantee your placement
within an Arcane College, grant you an apprenticeship with a famous
Archmage, or cause a magic-user who does not know you well to follow a
prescribed course of action based on your advice if the question is one
of Arcane knowledge.

\textbf{Dragonspeak.} Like the great wyrms of your ancestry, you have
practiced the ways of sneaking hidden charms and subliminal messages
within everyday conversation. Once per session, when you converse with
an intelligent creature for at least five minutes, you will learn one
useful secret of the GM's choosing about the creature.

\subsection{Dwarf}\label{dwarf}

\textbf{Warrior's Code.} As a veteran warrior, you command respect even
from foes. Once per session, you can use this ability to cause an enemy
or group of enemies to extend special concessions or favorable treatment
toward you via an unspoken warrior's code. The GM decides what these
concessions look like. For example, your enemies might choose to trust
you to come quietly and not shackle you, or overlook an insult that
would have otherwise have been cause for bloodshed.

\textbf{Stone Sense.} While underground you may fail to find what you're
looking for, but you can never be truly lost. You can always find your
way back to the entrance through which you entered.

\textbf{Dwarven Resilience}. Once per game session, you can
automatically succeed a \emph{Fortitude} action roll of Difficulty Level
equal or less than your Fortitude score.

\subsection{Elf}\label{elf}

\textbf{Elven Senses.} Your keen senses allow you to notice details that
others typically miss. Once per game session, you can use this ability
to notice something out of the ordinary. For example, you might spot a
hidden passage behind a bookcase, a trace of blood under the fingernails
of another character, or a wig that is not quite convincing. If you use
this ability after failing a \emph{Perception} roll, the GM decides
whether you notice the initial target of your roll or a different
detail.

\textbf{Sylvan Ally}. Creatures of nature can sense your deep respect
for the natural order. Wild animals give you a wide berth, Druids give
you the benefit of a doubt by assuming that you do not have destructive
intentions, and you can typically gain an audience with the chief of a
small local tribe by virtue of your race's reputation for defending
nature.

\subsection{Feytouched}\label{feytouched}

\textbf{Whisperer of the Wild.} Once per game session, you can ask a
single ``yes'' or ``no'' question of a plant or animal within earshot.
The plant or animal automatically trusts you at least enough to answer
the question truthfully. You receive the answer by way of an inner
sense, and so this ability cannot be used for further two-way
communication.

\textbf{Fey Innocence.} Your fey blood gives you a childlike quality
that can melt even the coldest of hearts. Once per game session, you can
leverage your fey innocence to turn an enemy and cause them to take pity
on you. The enemy might choose to look the other way when you've done
something illegal, forgive a debt you could never pay, or vouch in your
favor before the authorities.

\subsection{Half-Orc}\label{half-orc}

\textbf{Orcish Shakedown}. While others might convince with a silver
tongue, you speak the universal language of fear. Once per game session,
if you make a show of physical force, you can use your \emph{Might}
attribute for a \emph{Persuasion} roll. If your \emph{Persuasion} score
is already equal to or greater than your \emph{Might} score, you get
Advantage 1 on the roll.

\textbf{Orcish Scent.} Your sense of smell is similar to that of a wild
beast. As a focus action, you can discern the number and relative
location of living creatures within 60'. With an additional focus action
you can lock onto a particular scent and maintain its relative location
as long as it remains within 60'.

\subsection{Halfling}\label{halfling}

\textbf{Halfling Luck.} Once per game session, in a moment of need, you
can call on luck shine to upon you. The GM decides what form this luck
takes. For example, an attack that was meant for you might target an
ally instead, you may discover a secret passage to escape from a rolling
boulder, or a town guardsman decides to overlook your crime because you
happen to have grown up on the same street.

\textbf{Halfling Courage.} You have a brave heart for such a small
person. Once per game session, as a free action you can cancel the
effects of any bane relating to fear or a penalty associated with
negative morale.

\subsection{Human}\label{human}

\textbf{Human Nobility.} Being of high birth, you are treated as a
benefactor by the lower classes. They will trust and help you in the
hopes of being rewarded for their efforts. You are also treated as a
peer by lesser nobles and can typically request an audience with them.
In addition, representatives of the law generally assume you to be
beyond reproach unless they are presented with compelling evidence to
the contrary.

\textbf{Human Learning}. You have a knack for picking up new skills.
Once per game session, provided you are not under pressure from an
inordinately tight deadline, you can automatically succeed at a
non-attack action roll that relates to some craft, trade, skill, or
similar work provided it is Difficulty Level 2 or less.

\section{Step 5: Choose Starting
Equipment}\label{step-5-choose-starting-equipment}

In a typical game of \emph{Open Legend}, your character will start with
the gear he needs for the basic adventuring life. The GM, however, may
decide that the campaign starts under special circumstances (such as the
entire party caged in a slave convoy) that might dictate otherwise.

Usually, though, you begin with a Wealth Score of 2, and may select up
to three items of Wealth Level 2 and any number of items of a lesser
Wealth Level. See chapter 4 for rules concerning Wealth as well as
equipment details.

Instead of purchasing equipment a la carte, you may also elect to choose
one of the following starting packages built for common character
archetypes. After selecting your equipment, be sure to note any changes
to your defenses or other statistics.

\begin{longtable}[c]{@{}l@{}}
\toprule
Archetype Starting Packages\tabularnewline
\midrule
\endhead
\tabularnewline
\bottomrule
\end{longtable}

\begin{longtable}[c]{@{}lll@{}}
\toprule
Barbarian & Ranger & Monk\tabularnewline
\midrule
\endhead
Maul & Longbow & Quarterstaff\tabularnewline
Hatchet & Dagger & Hatchet\tabularnewline
Large Shield & Longsword & Leather Armor\tabularnewline
Long Bow & Chain Shirt &\tabularnewline
Battle Axe & Adventurer's Pack &\tabularnewline
Adventurer's Pack & &\tabularnewline
\bottomrule
\end{longtable}

\begin{longtable}[c]{@{}lll@{}}
\toprule
Paladin & Battle Mage & Mind Mage\tabularnewline
\midrule
\endhead
Warhammer & Crossbow & Longbow\tabularnewline
Short Bow & Quarterstaff & Quarterstaff\tabularnewline
Great Sword & Dagger & Dagger\tabularnewline
Large Shield & Leather Armor & Leather Armor\tabularnewline
Scale Mail & Mage's Pack & Mage's Pack\tabularnewline
Adventurer's Pack & &\tabularnewline
\bottomrule
\end{longtable}

\begin{longtable}[c]{@{}lll@{}}
\toprule
Assassin & Cleric & Druid\tabularnewline
\midrule
\endhead
5 Daggers & Flail & Quarterstaff\tabularnewline
Shortbow & Scale Shirt & Longbow\tabularnewline
Hand Crossbow & Large Shield & Dagger\tabularnewline
Leather Armor & Short Bow & Scale Shirt\tabularnewline
Rogue's Pack & Healer's Pack & Healer's Pack\tabularnewline
\bottomrule
\end{longtable}

\begin{longtable}[c]{@{}lll@{}}
\toprule
Shadowdancer & Bard & Arcane Protector\tabularnewline
\midrule
\endhead
5 Daggers & Dagger & Short Bow\tabularnewline
Shortbow & Longbow & Quarterstaff\tabularnewline
Hand Crossbow & Longsword & Dagger\tabularnewline
Leather Armor & Leather Armor & Leather Armor\tabularnewline
Rogue's Pack & Adventurer's Pack & Mage's Pack\tabularnewline
\bottomrule
\end{longtable}

\section{Step 6: Describe Your
Character}\label{step-6-describe-your-character}

\emph{Open Legend }is a role playing game, which means your character
will need more depth than attributes, feats, and gear. To make your
character come to life, add the following details. If you can't think of
anything yet, try to fill in the blanks during your first couple of play
sessions as you get to know your character better.

\textbf{A heroic name.} Be sure to check with your GM to see if he has
any particular setting in mind. Phil the Fighter would feel quite out of
place next to Therilas Windcaster and Gorion Skullcleaver.

\textbf{Two exceptional physical traits.} Think of the first two
features that other characters notice when they see you. Do your eyes
glow red when you are angry? Are you seven feet tall? Is your hair a
rainbow hue?

\textbf{Two defining social traits. }Maybe you stutter when you're
nervous. Maybe you don't trust anyone until they've proven themselves to
you. Or, perhaps, you are a winsome bard who almost always talks in
sing-song. Your two social traits should be characteristics that others
will learn shortly after getting to know you.

\textbf{A secret.} Your secret is something that other characters
probably won't find out about until they've gotten to know you quite
well. It's also a seed for great adventure that the GM can weave into
his campaign. Here are a few examples of character secrets to give you
some ideas:

\emph{Before Volkor changed his name and began wandering the land as a
barbarian sellsword, he was heir to the throne.}

\emph{Sir Thomas Tuckburrough served as an assassin for the local
thieves guild until a job went bad and he murdered an innocent
child--that's when he began his road to the priesthood.}

\emph{Talia was raised as a Druid of the Briar Rose, but she fled the
Order out of distaste for their violent ways. Now, she fears the
reprisal of her ex-brethren at every turn.}

\subsection{Tell Your Story}\label{tell-your-story}

With your character created, you are all ready to start playing
\emph{Open Legend}. Whether you're playing with old friends or complete
strangers, and whether you're completely new to roleplaying games or an
experienced veteran, the following tips will help ensure a fun time for
everyone at the table.

\subsection{Relax}\label{relax}

\emph{Open Legend }gives you a chance to step out of everyday life for a
few hours and into a fantastical world where you can perform heroic
deeds. Pour the Mountain Dew or grab a beer, order some takeout or pop
open the pretzels--but whatever you do, shake the dice like your life
depends on it and have fun. If you're playing a dwarf, maybe pull out
your best Scottish accent. If your character's a witch, squint your eyes
and speak in riddles. If you're no expert thespian, think of other ways
to add to the fun: play adventurous music on your phone, illustrate the
party's escapades, and so on.

\subsection{Respect the GM}\label{respect-the-gm}

If you've never GM'd before, you might not realize all the work that
goes into it. More likely than not, your GM worked for hours to put her
campaign together and prep for this session. Go with her storyline,
overlook any accidental inconsistencies, and don't cause a ruckus just
for the sake of causing a ruckus. If there's a dispute over the rules,
accept the GM's final ruling and agree to look it up later for the sake
of the game.

\subsection{Respect the Other Players}\label{respect-the-other-players}

Different people play roleplaying games for different reasons. Some
enjoy the tactical, chess-like combat encounters. Others just want to
tell an epic story. Still others are born actors, reveling in every
conversation with every character. Whatever it is that you enjoy about
playing \emph{Open Legend}, just remember that not everyone else at the
table may enjoy the same aspects. Part of the GM's duty is to give
everyone a chance to shine, but you can do your part too by not hogging
the spotlight and by encouraging the other players to have fun, whatever
that means for them.

\section{Beyond First Level}\label{beyond-first-level}

As the legend you are creating unfolds and grows in danger and
magnitude, your character's power will grow to match the challenge. This
gain in power is called leveling up. The GM decides when characters
level up. When your character does level up, you'll have a few decisions
to make.

\subsection{New Attribute Points}\label{new-attribute-points}

Upon leveling up, you gain 9 new attribute points to spend. You can use
these to increase your current attributes or buy completely new ones.
The cost to increase an attribute is equal to the new score. So, for
example, to raise your Might from 3 to 4 would cost 4 attribute points.
The cost to purchase a brand new attribute is summarized in the
Attribute Overview Table.

Until you reach 5th level, the maximum attribute score is 5. From levels
6 to 9, the maximum is equal to your level.

You do not have to use all of your attribute points at once, any
remaining attribute points can be saved for use at future levels.

\begin{longtable}[c]{@{}l@{}}
\toprule
Attribute Overview\tabularnewline
\midrule
\endhead
\tabularnewline
\bottomrule
\end{longtable}

\begin{longtable}[c]{@{}lll@{}}
\toprule
Attribute Score & Cost & Attribute Dice\tabularnewline
\midrule
\endhead
1 & 1 & 1d4\tabularnewline
2 & 3 & 1d6\tabularnewline
3 & 6 & 1d8\tabularnewline
4 & 10 & 1d10\tabularnewline
5 & 15 & 2d6\tabularnewline
6 & 21 & 2d8\tabularnewline
7 & 28 & 2d10\tabularnewline
8 & 36 & 3d8\tabularnewline
9 & 45 & 3d10\tabularnewline
\bottomrule
\end{longtable}

\subsection{New Feats}\label{new-feats}

Whenever you gain a level, you gain 3 feat points to purchase new feats.
You do not have to use all of these at once, and any remaining feat
points can be saved for future levels.

See \href{http://www.openlegendrpg.com/feats}{\emph{Feats}} to view the
complete list of feats to choose from.

\subsection{New Hit Points}\label{new-hit-points}

In \emph{Open Legend}, attributes are the means by which your hit points
increase. If you want your character to be able to take more hits,
increase either your Fortitude, Presence, or Will attribute. As outlined
in the default hit point formula, you'll gain 2 hit points each time you
raise one of those attributes by one.

\chapter{Concerning Hereditary
Principalities}\label{concerning-hereditary-principalities}

I will leave out all discussion on republics, inasmuch as in another
place I have written of them at length, and will address myself only to
principalities. In doing so I will keep to the order indicated above,
and discuss how such principalities are to be ruled and preserved.

I say at once there are fewer difficulties in holding hereditary states,
and those long accustomed to the family of their prince, than new ones;
for it is sufficient only not to transgress the customs of his
ancestors, and to deal prudently with circumstances as they arise, for a
prince of average powers to maintain himself in his state, unless he be
deprived of it by some extraordinary and excessive force; and if he
should be so deprived of it, whenever anything sinister happens to the
usurper, he will regain it.

We have in Italy, for example, the Duke of Ferrara, who could not have
withstood the attacks of the Venetians in '84, nor those of Pope Julius
in '10, unless he had been long established in his dominions. For the
hereditary prince has less cause and less necessity to offend; hence it
happens that he will be more loved; and unless extraordinary vices cause
him to be hated, it is reasonable to expect that his subjects will be
naturally well disposed towards him; and in the antiquity and duration
of his rule the memories and motives that make for change are lost, for
one change always leaves the toothing for another.
\end{document}
